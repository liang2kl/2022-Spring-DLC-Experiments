\documentclass[a4paper]{article}
\usepackage{ctex}
\usepackage{enumitem}
\usepackage{fancyhdr}
\usepackage{amsmath}
\usepackage{parskip}
\usepackage{float}
\usepackage{listings}
\usepackage{hyperref}
\usepackage{xcolor}

\setlength{\parskip}{6pt}

\pagestyle{headings}

\begin{document}
\title{点亮数字人生}
\author{梁业升 2019010547(计03)}

\maketitle

% GitHub styles
\definecolor{keyword}{HTML}{CF222E}
\definecolor{comment}{HTML}{6E7781}
\definecolor{string}{HTML}{0A3069}

\lstset{
    commentstyle=\color{comment},
    keywordstyle=\color{keyword},
    stringstyle=\color{string},
    basicstyle=\ttfamily\small,
    breakatwhitespace=false,
    breaklines=true,
    captionpos=b,
    keepspaces=true,
    showspaces=false,
    showstringspaces=false,
    showtabs=false,
}

\section{实验内容}

\begin{itemize}
    \item 依次显示学号和实验室房间号(2019010547+9208)
    \item 使用时钟控制,每1s切换一次
    \item 支持复位
\end{itemize}

\section{实现及其原理}
\subsection{代码}

\begin{lstlisting}[language=vhdl]
library IEEE;
use IEEE.STD_LOGIC_1164.ALL;
use IEEE.STD_LOGIC_ARITH.ALL;
use IEEE.STD_LOGIC_UNSIGNED.ALL;

entity digital_7 is
    port(
        display: out std_logic_vector(0 to 6); -- 译码后的信号
        clk: in std_logic;                     -- 时钟信号
        rst: in std_logic                      -- 复位信号
    );
end digital_7;

architecture bhv of digital_7 is
    -- 时钟上升沿次数
    signal cnt: integer := 0;
    -- 当前显示第几个数字(共14个)
    signal index: std_logic_vector(3 downto 0) := "0000";
begin    
    process(clk, rst)
    begin
        -- 如果有复位信号,恢复到初始状态
        if (rst = '1') then
            cnt <= 0;
            index <= "0000";
        
        -- 否则,判断时钟信号
        elsif (clk 'event and clk = '1') then
            if (cnt < 1000000) then
                -- 如果还没达到 1s(1MHz)
                cnt <= cnt + 1; -- 增加时钟计数
            else
                -- 达到了 1s
                cnt <= 0; -- 复位计数
                
                if (index = "1101") then
                    -- 当前是最后一个
                    index <= "0000"; -- 复位
                else
                    -- 否则
                    index <= index + 1;  -- 下一个数字
                end if;
                
            end if;
        end if;
    end process;
    
    process(index)
    begin
        case index is
            when "0000" => display <= "1101101"; -- 2
            when "0001" => display <= "1111110"; -- 0
            when "0010" => display <= "0110000"; -- 1
            when "0011" => display <= "1110011"; -- 9
            when "0100" => display <= "1111110"; -- 0
            when "0101" => display <= "0110000"; -- 1
            when "0110" => display <= "1111110"; -- 0
            when "0111" => display <= "1011011"; -- 5
            when "1000" => display <= "0110011"; -- 4
            when "1001" => display <= "1110000"; -- 7
            when "1010" => display <= "1110011"; -- 9
            when "1011" => display <= "1101101"; -- 2
            when "1100" => display <= "1111110"; -- 0
            when "1101" => display <= "1111111"; -- 8
            when others => display <= "0000000";
        end case;
    end process;

end bhv;
\end{lstlisting}

\subsection{实现原理}

使用变量 \texttt{cnt} 记录时钟变化的次数,使用变量 \texttt{index} 记录当前显示的数字位。
在一个以时钟信号 \texttt{clk} 和复位信号 \texttt{rst} 为敏感信号的 process 中,判断复位信号和时钟信号,
当复位信号为1,或距离上次复位或更新过了1s后,改变 \texttt{index}。

在另一个以 \texttt{index} 为敏感信号的 process 中,根据 \texttt{index} 的值改变输出信号,
即可改变显示的数字。

\section{实验总结}

由于是第一次进行可编程器件实验,本次实验中大量时间花在了配置环境和学习语言上。
通过课本的介绍,在这个过程中遇到的问题都逐一解决了,没有遇到太多的障碍。

实际实现时,遇到的问题主要是对硬件语言机制的不熟悉导致错误的做法,比如在多个 process 中同时更改某个
signal 的值。通过在 Stackoverflow 等网站进行搜索解决了相应的问题。

\end{document}
