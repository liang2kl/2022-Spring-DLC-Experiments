\documentclass[a4paper]{article}
\usepackage{ctex}
\usepackage{enumitem}
\usepackage{multirow}
\usepackage{fancyhdr}

\pagestyle{headings}

\setlist[description]{leftmargin=\parindent,labelindent=\parindent}

\begin{document}
\title{与非门电路的测试}
\author{梁业升 2019010547(计03)}

\maketitle

\section{实验目的}

加深对与非门基本特性和主要参数的理解,掌握主要参数的测试方法。

\section{实验原理}

\subsection{CMOS与非门的平均延迟时间$t_{{pd}}$}

输入信号(方波脉冲)接到与非门的两个输入上,输出 $v_O$ 是与非门输出结果,是 $v_{I}$ 的反向波形,它们之间的延迟时间即为与非门的延迟时间。

\subsection{CMOS与非门的电压传输特性}

电压传输特性是指输出电压 $v_{O}$ 与输入电压 $v_{I}$ 的函数关系,可以从曲线上直接读出一些静态参数:

\begin{enumerate}
    \item 输出高电平 $V_{OH}$
    \item 输出低电平 $V_{OL}$
    \item 关门电平 $V_{OFF}$:使电路输出处于高电平状态所允许的最大输入电压
    \item 开门电平 $V_{ON}$:使电路输出处于低电平状态所允许的最小输入电压
    \item 高电平噪声容限电压 $V_{NH}=V_{OH_{min}}-V_{ON}$,表示输入为高电平时所允许噪声电压的最大值
    \item 低电平噪声容限电压 $V_{NL}=V_{OFF}-V_{OL_{max}}$,表示输入为低电平时所允许噪声电压的最大值
\end{enumerate}

\subsection{TTL与非门的平均延迟时间$t_{pd}$}

由于 TTL 与非门的平均延迟时间很小,需要将几个与非门串联起来测量总的平均延迟时间。

\subsection{TTL与非门的电压传输特性}

与 CMOS 与非门相同。

\section{实验内容及要求}

\begin{enumerate}
    \item 测量 CMOS 与非门 CD4011 的平均延迟时间,测量电路如课本图 4.1(b) 所示,其中输入电压 $v_1$ 可选择低电平为 0V,高电平为 5V,频率为 1MHz 的方波信号。
    \item 测量 CMOS 与非门 CD4011 的电压传输特性,测量电路如课本图 4.3 所示,其中输入电压 $v_1$ 可选择低电平为 0V,高电平为 5V,频率为 100Hz 的三角波信号。
    \item 测量 TTL 与非门 74LS00 的平均延迟时间 $t_{{pd}}$,测量电路如课本图 4.4(a) 所示,其中信号是信号发生器的TTL输出端产生的频率为2MHz的方波信号。
    \item 测量TTL与非门74LS00的电压传输特性,测量电路与输入信号与 2 相同。
\end{enumerate}

\section{思考题}

\textbf{应如何处理TTL与非门和CMOS与非门的多余输入端?}

应将其全部接地,使其有确定的电平。

\section{实验结果}

\begin{figure}
    \includegraphics[width = \textwidth]{img/1.jpeg}
    \caption{CMOS 与非门 CD4011 的平均延迟时间}
\end{figure}

\begin{figure}
    \includegraphics[width = \textwidth]{img/2.jpeg}
    \caption{CMOS 与非门 CD4011 的电压传输特性}
\end{figure}

\begin{figure}
    \includegraphics[width = \textwidth]{img/3.jpeg}
    \caption{TTL 与非门 74LS00 的平均延迟时间 $t_{{pd}}$}
\end{figure}

\begin{figure}
    \includegraphics[width = \textwidth]{img/4.jpeg}
    \caption{TTL与非门74LS00的电压传输特性}
\end{figure}


\end{document}
