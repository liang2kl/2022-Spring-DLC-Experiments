\documentclass[a4paper]{article}
\usepackage{ctex}
\usepackage{enumitem}
\usepackage{multirow}
\usepackage{fancyhdr}

\pagestyle{headings}

\setlist[description]{leftmargin=\parindent,labelindent=\parindent}

\begin{document}
\title{示波器实验}
\author{梁业升 2019010547(计03)}

\maketitle

\section{实验目的}

\begin{enumerate}
    \item 熟悉示波器的使用,包括探头校准,探头衰减常数设置,示波器的控制、测量与结果保存,以及内置信号发生器的使用。
    \item 学会生成并测量特定频率、幅度的正弦、TTL 方波、三角波等常见波形。
\end{enumerate}

\section{实验内容及要求}

用示波器生成下列三种波形:

\begin{enumerate}
    \item 100kHz 正弦波,占空比 50\%,直流电平为零,峰峰值 4V;
    \item 1MHz TTL 方波,占空比 50\%;
    \item 100Hz,0-5V 三角波,占空比 50\%。
\end{enumerate}

对于每一种波形,保存波形图并测量频率、低电平、高电平。

\section{实验步骤}

\subsection{探头校准}

打开电源,将示波器探头的钩子连接到 Demo2(探头补偿)端子上,探头的黑夹子连接到中间接地端子,依次按下 Auto Scale(自动调整)键、通道键、“探头”、“无源探头检查”。如果是过补偿和欠补偿 ,要使用专用工具调整探头上的微调电容,以获得尽可能平的脉冲。

\subsection{生成波形}

选取示波器的“Wave Gen”功能,由屏幕提示,用旋钮调节波形类型、频率、高电平、低电平、占空比即可。需要注意的是,最终的 4V、0-5V 等应以测量为准,而非以设定值为准。

\subsection{调整探头}

调整探头的衰减设置,并在示波器的探头参数设置中调整到相同的衰减比率。信号超过 40V 或频率超过 500 kHz 时要使用“×10”的探头,因此波形(1)、(3)使用“×1”探头,波形(2)使用“×10”探头。

\subsection{测量波形}

将探头的地线夹子与内置信号发生器输出端口的黑夹子相连,探头的输入信号与内置信号发生器输出端口的红夹子相连。使用示波器的“Auto Scale”按钮来自动定标波形图,使用示波器的“Meas”按钮,设置测量菜单里的相应参数,测量所需的参数。

\subsection{保存波形}

使用“Save/Recall”功能或者手机拍下屏幕, 保存波形图和波形数据。

\section{实验结果}

\begin{figure}
    \includegraphics[width = \textwidth]{img/1.jpeg}
    \caption{100kHz 正弦波,占空比 50\%,直流电平为零,峰峰值 4V}
\end{figure}

\begin{figure}
    \includegraphics[width = \textwidth]{img/2.jpeg}
    \caption{1MHz TTL 方波,占空比 50\%}
\end{figure}

\begin{figure}
    \includegraphics[width = \textwidth]{img/3.jpeg}
    \caption{100Hz,0-5V 三角波,占空比 50\%}
\end{figure}

\end{document}
